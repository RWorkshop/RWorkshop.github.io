

	\documentclass[a4paper,12pt]{article}
%%%%%%%%%%%%%%%%%%%%%%%%%%%%%%%%%%%%%%%%%%%%%%%%%%%%%%%%%%%%%%%%%%%%%%%%%%%%%%%%%%%%%%%%%%%%%%%%%%%%%%%%%%%%%%%%%%%%%%%%%%%%%%%%%%%%%%%%%%%%%%%%%%%%%%%%%%%%%%%%%%%%%%%%%%%%%%%%%%%%%%%%%%%%%%%%%%%%%%%%%%%%%%%%%%%%%%%%%%%%%%%%%%%%%%%%%%%%%%%%%%%%%%%%%%%%
\usepackage{eurosym}
\usepackage{vmargin}
\usepackage{amsmath}
\usepackage{graphics}
\usepackage{framed}
\usepackage{epsfig}
\usepackage{subfigure}
\usepackage{enumerate}
\usepackage{fancyhdr}

\setcounter{MaxMatrixCols}{10}
%TCIDATA{OutputFilter=LATEX.DLL}
%TCIDATA{Version=5.00.0.2570}
%TCIDATA{<META NAME="SaveForMode"CONTENT="1">}
%TCIDATA{LastRevised=Wednesday, February 23, 201113:24:34}
%TCIDATA{<META NAME="GraphicsSave" CONTENT="32">}
%TCIDATA{Language=American English}

\pagestyle{fancy}
\setmarginsrb{20mm}{0mm}{20mm}{25mm}{12mm}{11mm}{0mm}{11mm}
\lhead{Introduction to R} \rhead{Kevin O'Brien} \chead{Decision tree} %\input{tcilatex}

\begin{document}
\section*{Exercise:  The Air Quality Data Set}

A data frame with 154 observations on 6 variables.

\begin{description}
\item[Ozone]	numeric	Ozone (ppb)
\item[Solar.R]	numeric	Solar R (lang)
\item[Wind]	numeric	Wind (mph)
\item[Temp]	numeric	Temperature (degrees F)
\item[Month]	numeric	Month (1--12)
\item[Day]	numeric	Day of month (1--31)
\end{description}
%======================================= %
\begin{framed}
	\begin{verbatim}
	
	tail(airquality)
	help(airquality)
	
	\end{verbatim}
\end{framed}
%=======================================%

\subsection{Exercises}
\begin{enumerate}
	\item For each variable - how many missing values are there?
	\item How may complete cases are there? 
	\item What is the variance of each of the continuous variables?
	\item How many complete cases are there (i.e. no missing values)
	\item If you dont include the "wind" variable, how many complete cases are there?
\end{enumerate}
%=======================================%
\subsection{Complete Cases}
For a data frame of $n$ cases, the command \texttt{complete.cases()} returns an $n$ element logical vector.

Logical vectors ( \texttt{TRUE} and \texttt{FALSE} ) can be converted into the corresponding numeric values (1 and 0 respectively) by using the \texttt{as.numeric()} command.

\begin{framed}
	\begin{verbatim}
> X <- c(T,T,F,F,T)
>
> as.numeric(X)
[1] 1 1 0 0 1
>
	\end{verbatim}
\end{framed}
\end{document}
