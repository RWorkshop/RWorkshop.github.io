
Stem and Leaf Plots
Bivariate Data
Simple Scatterplots, Correlation and Covariance
X1 =
Y1 =
Plot(X1,Y1)
cor(X1)
cov(Y1)
Anscombe’s Quartet
The uniform distribution
 
 
The "apply" family of functions
The Apply family of functions are


Iteration Loops
Central Limit Theorem
Recall the Dice experiment in week 8.
 
N=100           	          	          	             #number of loops
Avgs=numeric(N)           	             #array “Avgs” store the sample means
for( i in 1:N)
              {              Dice=floor(runif(50,min=1,max=7));              Avgs[i]=mean(Dice);
              }
Avgs           	          	          	             #print Avgs dataset to screen
 
The Central limit theorem states that.
The “Dice” distribution is a discrete uniform distribution. However
mean(Avgs)           	          	             #compute the mean. Is it roughly what we are expecting?
qqnorm(Avgs)           	          	             #draws a QQ plot that is used to check for normality.
qqline(Avgs)           	          	             #adds trend line to QQplot.
shapiro.test(Avgs)           	             #Shapiro Wilk test. Normality is assumed if p-value > 0.05.
           	          	            

Gambler’s Ruin
In anticipation of future material on course MS4217.

Markov Chains
In anticipation of future material on course MS4217.
For the purposes of this course, we will assume the initial state is state 1.
#Transition matrix “Tron”
Tron = matrix (c(0.5,0.25,0.25,0.4,0.3,0.3,0.2,0.7,0.1),nrow=3,byrow=TRUE)
rownames(Tron)=c("state1","state2","state3")
colnames(Tron) = rownames(Tron)
 
Assume that the process is currently in state 1. The uniform distribution can be used to compute a random value for probability.
> runif(1)           	                           #return one random value between 0 and 1
[1] 0.6891122
The outcome is between 0.5 and 0.75, therefore the process will migrate to state 2. If the probability value was less than 0.5, the process would have remained in state 1. If the probability was greater than 0.75, the  process would have migrated to state 3.
 
Tron = matrix (c(0.5,0.25,0.25,0.4,0.3,0.3,0.2,0.7,0.1),nrow=3,byrow=TRUE)
#check that each row totals is one.
sum( Tron[1,]) == 1;                sum( Tron[2,]) == 1 ;               sum( Tron[3,]) ==1 ;
#compute upper and lower ‘fences’ for each row.
L1 =Tron[1,1];   	             U1 = Tron[1,1] + Tron[1,2];
L2 =Tron[2,1];   	             U2 = Tron[2,1] + Tron[2,2];
L3 =Tron[3,1];            	             U3 = Tron[3,1] + Tron[3,2];
# inital conditions
State = 1           	             #initial state
N=4000           	          	             #number of loops
Migrn = numeric(N)              #array that can be used to keep track of migrations
Visits=numeric (3)              #array that counts the number of visits to each state

 
for ( i in 1:N)
{
Migrn[i] = State           	          	             #record the state for phase i
Visits[State] = Visits[State]+1           	             #update the number of visits made to each state
Prob = runif(1)
if (State==1){
             	       if (Prob <= L1){State = 1};if (Prob > L1) {State = 2};if (Prob > U1) {State = 3};
           	 	 }
if (State==2){
             	       if (Prob <= L2){State = 1};if (Prob > L2) {State = 2};if (Prob > U2) {State = 3};
           	 	          	      }
if (State==3){
             	       if (Prob <= L3){State = 3};if (Prob > L3) {State = 2};if (Prob > U3) {State = 3};
           	 	          	      }
}              # complete the "for" loop
Migrn           	          	            
Visits           	          	             #how many visits were made to each state?
 
