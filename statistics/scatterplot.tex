%=====================================================================%
\begin{frame}[fragile]
\frametitle{Scatterplot}
One of the first steps in a regression analysis is to determine if any
kind of relationship exists between the two variables.
A scatterplot is created and can initially be used to get an idea
about the nature of the relationship between the variables, e.g. if
the relationship is linear, curvilinear, or no relationship exists.
Read the data from the file SLRex1.txt into a data frame called
SLR1.
\end{frame}
%=====================================================================%
\begin{frame}[fragile]
\frametitle{Scatterplot}
SLR1 <- read.table("C:/.../SLRex1.txt", header=T)
SLR1
x y
1 5.366516 26.76595
2 6.435778 46.89376
3 7.831232 34.11415
4 7.587142 45.49667
5 5.380939 33.22162
6 8.254098 39.98920
...

\end{frame}
%=====================================================================%
\begin{frame}[fragile]
\frametitle{Scatterplot}
This file contains readings for a response variable y and an
explanatory variable x.
To create a scatterplot in R, simply use the plot command.
Remember y is the response variable and x is the explanatory
variable:
plot(x=SLR1$x, y=SLR1$y, main="Scatterplot")
OR
attach(SLR1)
plot(x=x, y=y, main="Scatterplot")
\end{frame}
%=====================================================================%
\begin{frame}[fragile]
\frametitle{Scatterplot}
\begin{itemize}
\item (The attach command makes the variables in the data frame
SLR1 accessible by name rather than using the $ notation). 
\item Need
to be careful as if there are other variables in use with the same
names as those in the data frame, R will get confused. 
\item When
finished working with SLR1, need to use detach(SLR1).
\end{itemize}
\end{frame}
%=====================================================================%
\begin{frame}[fragile]
\frametitle{Scatterplot}
There is clearly a positive linear relationship between x and y. As x
increases y also increases. No obvious outliers. SLR is appropriate
here.
\end{frame}
%=====================================================================%
\end{document}
